%Plantilla anteproyecto TFG
%Última modificación: 28 de mayo de 2021
\documentclass[12pt,oneside,a4paper]{article}
\usepackage[spanish]{babel}
\usepackage[utf8]{inputenc}
\usepackage{graphicx}
\usepackage{amsmath}
\usepackage{amssymb}
\usepackage{xcolor}
\usepackage{subfigure}
\usepackage{url}
\linespread{1}
\setlength{\parskip}{1\baselineskip}
\parindent 1cm
\sloppy


%Opciones que debes descomentar mientras estemos revisando el anteproyecto
\usepackage{lineno}
\linenumbers


\usepackage[pagebackref=true,breaklinks=true,letterpaper=true,colorlinks,bookmarks=true]{hyperref}


%lista de palabras que Latex no parte bien
\hyphenation{pa-la-bras lis-ta}

\begin{document}

\thispagestyle{empty}

\begin{center}

\begin{large}
Departamento de Teoría de la Señal y Comunicaciones\\
Escuela Politécnica Superior\\
\end{large}
\vspace{1cm}

\includegraphics[width=8cm]{figuras/logo-uah.pdf}

\textbf{ANTEPROYECTO}

\vspace{1cm}

\begin{large}\textbf{\textit{Título del proyecto}}\end{large}

\vfill

Mayo - 2021

\end{center}

\begin{flushright}
\textit{Autor - \textbf{Tu nombre}} \\
\textit{Director - \textbf{El de tu director}}
\end{flushright}

\newpage

\section{Introducción}

Escribe una breve introducción para el anteproyecto, destacando:
\begin{itemize}
 \item el título del proyecto y el objetivo principal que persigues.
 \item referencias a otros trabajos similares que encuentres en la literatura (y por supuesto en otros trabajos fin de grado del departamento).
 \item Trata de responder a la pregunta: ¿qué voy a hacer?, NO ¿cómo lo voy a hacer? Los detalles sobre el cómo se van a hacer las cosas los daremos después.
\end{itemize}

\subsection{Estado del arte}
En esta subsección es donde puedes hablar del resto de trabajos y poner tantas referencias como quieras (e.g. \cite{bibtex}).

\section{Objetivos y campos de aplicación}
Destaca los objetivos particulares, y también los campos de aplicación (haz una lista con las aplicaciones reales de tu trabajo).

\section{Descripción del trabajo}
Ahora si, ¿cómo vas a hacer las cosas? Describe las partes, módulos o fases de tu proyecto, y añade un diagrama de bloques. Comenta cada uno de ellos, para que el lector entienda lo que persigues.

\section{Fases de desarrollo}
Ahora debes detallar, de forma técnica, las fases de desarrollo. Te pongo un ejemplo.

\begin{enumerate}
\item Estudio de bibliografía y documentación sobre el problema de caza de pesca con gusano. Duración: 2 semanas.
\item \ldots
\item Edición del manual de usuario de la aplicación. Duración: 1 semana.
\item Edición del documento final utilizando \LaTeX. Duración: 1 mes.
\end{enumerate}

\textcolor{red}{IMPORTANTE: debes indicar en cada item la duración de esa fase del proyecto}.

\section{Medios disponibles}

Describe los medios que vas a emplear para realizar el TFG: tipo de ordenador, software, SO, librerías, acceso a servidores de cálculo, plataformas robóticas concretas, bases de datos, etc.

\section{Sobre \LaTeX}

Si ya manejas \LaTeX no hace falta que leas esta sección.

Necesitas aprender a manejar \LaTeX para escribir este anteproyecto y la memoria final del TFG. Verás que se trata de una herramienta estupenda para editar textos de forma profesional.

La mejor manera de aprender es abrir un fichero \verb$.tex$ (por ejemplo el \verb$anteproyecto.tex$) y empezar a cambiar cosas en él y a compilarlo para generar un fichero \verb$.pdf$. Como editor te recomendamos TexMaker \cite{texmaker} (está en los repositorios de Ubuntu).

Un página estupenda donde consultar dudas es la siguiente \cite{wikibook}.

Ya has visto que las citas bibliográficas (referencias a páginas web, artículos, congresos y libros) debes ponerlas utilizando el comando \verb$\cite$.

Toda la bibliografía debes tenerla en un fichero \verb$.bib$ (en este caso estamos utilizando el fichero \verb$bibliografia-tfc.bib$). Se trata de un fichero que tiene la bibliografía en formato BibTex \cite{bibtex}. Para gestionar este fichero te recomendamos utilizar el programa JabRef \cite{jabref}. Si abres con JabRef el fichero \verb$bibliografia-tfc.bib$ verás las referencias que hemos utilizado en esta plantilla de anteproyecto. Tienes referencias a páginas web (que debes poner como tipo MISC) \cite{bibtex,jabref,wikibook}, a artículos de congresos (tipo INPROCEEDINGS) \cite{Lowe1999}, a artículos en revistas (tipo ARTICLE) \cite{Tuytelaars2008} y a libros (tipo BOOK) \cite{hartley2006}.

Como ves, \LaTeX se encarga de numerar las referencias y de generar la bibliografía a partir del fichero \verb$.bib$.

También puedes poner imágenes, como la representada en la Figura \ref{fig:ejemplo}. Recuerda utilizar \textbf{SÓLO} el formato \verb$.pdf$ y compilar con pdflatex. Para generar figuras en formato \verb$.pdf$ te recomiendo utilizar el programa Inkscape\footnote{\url{http://www.inkscape.org/} -- también está en los repositorios de Ubuntu}. Por cierto, las notas a pie de página como ésta debes tratar de no utilizarlas en exceso cuando escribas tu proyecto.

\begin{figure}
  \centering
  \includegraphics[width=4cm]{figuras/logo-uah.pdf}
  \caption{Un título que describa el contenido de la figura.}
  \label{fig:ejemplo}
\end{figure}


%Bibliografía
\bibliographystyle{plain}
\bibliography{bibliografia-tfc}


\end{document}
